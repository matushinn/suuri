\documentclass{jsarticle}
\begin{document}
\title{情報数理学習レポート}
\author{大江祥太郎}

\section{目的}
自律移動ロボット(ANIMABOT BEATLE)を使って、ソフトウェア制御を実践し、可動部を持つハードウェアには必ずばらつきが存在することを理解し、
ロボットを正確に動かすための制御方法を考察する。

\section{方法}
ANIMABOT ROBOT(自律移動ロボット)を利用する。\\
このロボットに内蔵されたプロセッサのタイマ-を使用することにより、BEATLEの動作する時間を制御できる。\\
またこのロボットの制御プログラムは、PC上の開発環境においてC言語により作成する。\\
作成したプログラムは、実行可能なモジュールに変換されBEATLE本体に転送され、ロボット上の処理プロセッサで実行され、
ロボットが動作を行う。\\
また可動部を持つハードウェアには誤差があり、全く同じ動作を繰り返させても、毎回違いが生じる繰り返しのばらつきと、部品が設計通りになっていないため、カタログ値とズレが生じる系統的なズレがある。\\
このようなばらつきはコストをかければ減らすことはできるが、それは使用上の要求とコストのバランスで決められる。\\

\section{考察}
\subsection{実験1による考察}
BEATLEを前進させるプログラム
  \begin{verbatim}
  #include <Servo.h>//モータ用のインクルード
    Servo servoL;
    Servo servoR;
    Servo arm;
    int a; // arm sensor

    void setup() {
      // put your setup code here, to run once:
    servoR.attach(4);//右モータのピンが4に配線されている場合のアタッチ
    servoL.attach(5);//左モータのピンが5に配線されている場合のアタッチ
    arm.attach(8);//8番のモータをアームのモータとする
    /*モータ回転位置初期化*/
    servoR.write(90);
    servoL.write(90);
    arm.write(90);
    Serial.begin(9600);//シリアル通信の転送速度
    }

    void set_motor(int L,int R){
      servoL.write(L);
      servoR.write(R);
    }

    void loop() {
    /*ここにプログラムを記述する*/
    set_motor(150,30);//前進
     delay(2000);
    set_motor(90,90);//停止
     delay(5000);

    }
  \end{verbatim}
\begin{table}[htbp]
  \begin{center}
      \caption{実験1のグラフ}
          \begin{tabular}{c|c|c|c|c|c}
            \hline
            X & Y(1回目) & Y(2回目) & Y(3回目) & Y(4回目) & Y(5回目) \\ \hline \hline
            1.0 & 55.2 & 59.5 & 59.2 & 61.4 & 62.1 \\
            2.0 & 137.4 & 134.6 & 136.9 & 134.5 & 139.8 \\
            3.0 & 205.5 & 203.4 & 200.4 & 200.2 & 199.5 \\
            4.0 & 276.5 & 275.1 & 272.4 & 271.8 & 271.5  \\
            \hline
          \end{tabular}
  \end{center}
\end{table}


\subsection{実験2による考察}
BEATLEを方向転換させるプログラム
  \begin{verbatim}
  #include <Servo.h>//モータ用のインクルード
    Servo servoL;
    Servo servoR;
    Servo arm;
    int a; // arm sensor

    void setup() {
      // put your setup code here, to run once:
    servoR.attach(4);//右モータのピンが4に配線されている場合のアタッチ
    servoL.attach(5);//左モータのピンが5に配線されている場合のアタッチ
    arm.attach(8);//8番のモータをアームのモータとする
    /*モータ回転位置初期化*/
    servoR.write(90);
    servoL.write(90);
    arm.write(90);
    Serial.begin(9600);//シリアル通信の転送速度
    }

    void set_motor(int L,int R){
      servoL.write(L);
      servoR.write(R);
    }

    void loop() {
    /*ここにプログラムを記述する*/
    set_motor(150,150);//右旋回
      delay(3000);

    set_motor(91,91);//停止
     delay(10000);

    }
  \end{verbatim}
\begin{table}[htbp]
  \begin{center}
      \caption{実験2のグラフ}
          \begin{tabular}{c|c|c|c|c|c}
            \hline
            X & Y(1回目) & Y(2回目) & Y(3回目) & Y(4回目) & Y(5回目) \\ \hline \hline
            1.0 & 63.5 & 63.5 & 63.5 & 63.5 & 63.5 \\
            2.0 & 124.5 & 126.5 & 127.5 & 126.0 & 125.0 \\
            3.0 & 264.0 & 263.5 & 265.0 & 260.5 & 263.5 \\
            4.0 & 389.5 & 390.5 & 388.0 & 388.5 & 388.0  \\
            \hline
          \end{tabular}
  \end{center}
\end{table}

\subsection{実験3による考察}
実験3のプログラム
  \begin{verbatim}
  #include <Servo.h>//モータ用のインクルード
  Servo servoL;
  Servo servoR;
  Servo arm;
  int a; // arm sensor

  void setup() {
    // put your setup code here, to run once:
  servoR.attach(6);//右モータのピンが4に配線されている場合のアタッチ
  servoL.attach(7);//左モータのピンが5に配線されている場合のアタッチ
  arm.attach(8);//8番のモータをアームのモータとする
  /*モータ回転位置初期化*/
  servoR.write(90);
  servoL.write(90);
  arm.write(90);
  Serial.begin(9600);//シリアル通信の転送速度
  }

  void set_motor(int L,int R){
    servoL.write(L);
    servoR.write(R);
  }

  void loop() {
  /*ここにプログラムを記述する*/
  set_motor(120,50);//直進1
  delay(4000);
  set_motor(90,90);//停止
   delay(500);


  set_motor(168,143);//右旋回 90°
    delay(800);
  set_motor(91,91);//停止
   delay(500);


  set_motor(117,50);//直進2
    delay(2500);//2秒プログラム止めて直進を維持
  set_motor(91,91);//停止
   delay(1000);


  set_motor(168,143);//右旋回 90°
    delay(750);
  set_motor(90,90);//停止
   delay(1000);


  set_motor(117,50);//直進3
    delay(3700);
  set_motor(91,91);//停止
   delay(1000);


  set_motor(48,23);//左旋回 90
    delay(950);
  set_motor(91,91);//停止
   delay(1000);


  set_motor(117,50);//直進4
    delay(2450);
  set_motor(91,91);//停止
   delay(1000);


  set_motor(48,23);//左旋回 90  -400
    delay(950);
  set_motor(91,91);//停止
   delay(1000);


  set_motor(117,50);//直進5
    delay(1700);
  set_motor(91,91);//停止
   delay(1000);


  set_motor(168,143);//右旋回 90°
    delay(750);
  set_motor(91,91);//停止
   delay(1000);


  set_motor(117,50);//直進
    delay(3700);
  set_motor(91,91);//停止
   delay(1000);


    set_motor(48,23);//左旋回
    delay(950);// 90
  set_motor(91,91);//停止
   delay(1000);


  set_motor(117,50);//直進
    delay(1600);
  set_motor(91,91);//停止
   delay(1000);


  set_motor(91,91);//停止
   delay(10000);

  }

\end{table}

\subsection{実験4による考察}
\begin{table}[htbp]
  \begin{center}
  \caption{PCのグラフ}
      \begin{tabular}{c|c|c|c|c}
        \hline
        X & Y(1回目) & Y(2回目) & Y(3回目) & Y(4回目)\\ \hline \hline
        90mm & 1 & 2 & 1 & 2 \\
        100mm & 462 & 457 & 465 & 462 \\
        120mm & 637 & 638 & 637 & 639 \\
        150mm & 635 & 636 & 635 & 635\\
        200mm & 413 & 412 & 415 & 414 \\
        \hline
      \end{tabular}
    \end{center}
\end{table}
begin{table}[htbp]
  \begin{center}
  \caption{PCケース(布)のグラフ}
      \begin{tabular}{c|c|c|c|c}
        \hline
        X & Y(1回目) & Y(2回目) & Y(3回目) & Y(4回目)\\ \hline \hline
        90mm & 380 & 378 & 379 & 382 \\
        100mm & 470 & 469 & 466 & 467 \\
        120mm & 580 & 583 & 579 & 578 \\
        150mm & 640 & 689 & 642 & 640\\
        200mm & 439 & 438 & 437 & 431 \\
        \hline
      \end{tabular}
    \end{center}
\end{table}
begin{table}[htbp]
  \begin{center}
  \caption{箱(ダンボール)のグラフ}
      \begin{tabular}{c|c|c|c|c}
        \hline
        X & Y(1回目) & Y(2回目) & Y(3回目) & Y(4回目)\\ \hline \hline
        90mm & 378 & 377 & 375 & 376 \\
        100mm & 461 & 464 & 461 & 462 \\
        120mm & 605 & 608 & 607 & 606 \\
        150mm & 639 & 641 & 640 & 641\\
        200mm & 426 & 419 & 421 & 422 \\
        \hline
      \end{tabular}
    \end{center}
\end{table}
begin{table}[htbp]
  \begin{center}
  \caption{ブロック(コンクリート)のグラフ}
      \begin{tabular}{c|c|c|c|c}
        \hline
        X & Y(1回目) & Y(2回目) & Y(3回目) & Y(4回目)\\ \hline \hline
        90mm & 369 & 370 & 365 & 366 \\
        100mm & 279 & 280 & 282 & 281 \\
        120mm & 617 & 615 & 618 & 616 \\
        150mm & 635 & 633 & 634 & 636 \\
        200mm & 414 & 413 & 411 & 412 \\
        \hline
      \end{tabular}
    \end{center}
\end{table}

\iffalse
\begin{table}[htbp]
  \begin{center}
      \caption{PCのグラフ}
          \begin{tabular}{c|c|c|c|c}
            \hline
            X & Y(1回目) & Y(2回目) & Y(3回目) & Y(4回目)\\ \hline \hline
            90mm & 1 & 2 & 1 & 2 \\
            100mm & 462 & 457 & 465 & 462 \\
            120mm & 637 & 638 & 637 & 639 \\
            150mm & 635 & 636 & 635 & 635\\
            200mm & 413 & 412 & 415 & 414 \\
            \hline
          \end{tabular}
        \end{center}
        \caption{PCケース(布)のグラフ}
            \begin{tabular}{c|c|c|c|c}
              \hline
              X & Y(1回目) & Y(2回目) & Y(3回目) & Y(4回目)\\ \hline \hline
              90mm & 380 & 378 & 379 & 382 \\
              100mm & 470 & 469 & 466 & 467 \\
              120mm & 580 & 583 & 579 & 578 \\
              150mm & 640 & 689 & 642 & 640\\
              200mm & 439 & 438 & 437 & 431 \\
              \hline
            \end{tabular}
          \end{center}
          \caption{箱(ダンボール)のグラフ}
              \begin{tabular}{c|c|c|c|c}
                \hline
                X & Y(1回目) & Y(2回目) & Y(3回目) & Y(4回目)\\ \hline \hline
                90mm & 378 & 377 & 375 & 376 \\
                100mm & 461 & 464 & 461 & 462 \\
                120mm & 605 & 608 & 607 & 606 \\
                150mm & 639 & 641 & 640 & 641\\
                200mm & 426 & 419 & 421 & 422 \\
                \hline
              \end{tabular}
            \end{center}
            \caption{ブロック(コンクリート)のグラフ}
                \begin{tabular}{c|c|c|c|c}
                  \hline
                  X & Y(1回目) & Y(2回目) & Y(3回目) & Y(4回目)\\ \hline \hline
                  90mm & 369 & 370 & 365 & 366 \\
                  100mm & 279 & 280 & 282 & 281 \\
                  120mm & 617 & 615 & 618 & 616 \\
                  150mm & 635 & 633 & 634 & 636 \\
                  200mm & 414 & 413 & 411 & 412 \\
                  \hline
                \end{tabular}
              \end{center}
    \end{tabular}
  \end{center}
\end{table}
\fi

\section{実験5による考察}
BEATLEを前進させ、障害物を感知したら90度方向転換させるプログラム
  \begin{verbatim}
  #include <Servo.h>//モータ用のインクルード
    Servo servoL;
    Servo servoR;
    Servo arm;
    int a; // arm sensor

    void setup() {
      // put your setup code here, to run once:
    servoR.attach(4);//右モータのピンが4に配線されている場合のアタッチ
    servoL.attach(5);//左モータのピンが5に配線されている場合のアタッチ
    arm.attach(8);//8番のモータをアームのモータとする
    /*モータ回転位置初期化*/
    servoR.write(90);
    servoL.write(90);
    arm.write(90);
    Serial.begin(9600);//シリアル通信の転送速度
    }

    /*このプログラムではモータの回転角(速度)を与える関数を作っている*/
    void set_motor(int L, int R){
      servoL.write(L);//0≦L≦180、90から離れるほど速く回転
      servoR.write(R);//0≦R≦180
    }

    void loop() {
    /*ここにプログラムを記述する*/
    a = analogRead(A5);//A5番のピンについている距離センサの値を読み取り
      Serial.println(a);//センサ値出力
      Serial.println();//シリアルモニタ上で改行
    if(a>500) {//回転
      arm.write(90);//アームが前向きになるよう維持
      set_motor(150,150);//右旋回
      delay(750);//n秒の時n*1000の値を入力
    }else{//それ以外のとき直進するように速度をあたえる
      set_motor(160,56);//直進
      arm.write(90);//アームが前向きになるよう維持
    }
      delay(100);
    }

  \end{verbatim}

\section{実験6による考察}
  オプション課題のプログラム
    \begin{verbatim}
    #include <Servo.h>//モータ用のインクルード
      Servo servoL;
      Servo servoR;
      Servo arm;
      int a; // arm sensor

      void setup() {
        // put your setup code here, to run once:
      servoR.attach(6);//右モータのピンが4に配線されている場合のアタッチ
      servoL.attach(7);//左モータのピンが5に配線されている場合のアタッチ
      arm.attach(8);//8番のモータをアームのモータとする
      /*モータ回転位置初期化*/
      servoR.write(90);
      servoL.write(90);
      arm.write(90);
      Serial.begin(9600);//シリアル通信の転送速度
      }

      /*このプログラムではモータの回転角(速度)を与える関数を作っている*/
      void set_motor(int L, int R){
        servoL.write(L);//0≦L≦180、90から離れるほど速く回転
        servoR.write(R);//0≦R≦180
      }

      void loop() {
      /*ここにプログラムを記述する*/
      a = analogRead(A5);//A5番のピンについている距離センサの値を読み取り
        Serial.println(a);//センサ値出力
        Serial.println();//シリアルモニタ上で改行
      if(a>600) {//回転
        arm.write(90);//アームが前向きになるよう維持

        set_motor(120,50);//直進
        delay(300);//n秒の時n*1000の値を入力

        arm.write(90);//アームが前向きになるよう維持

        set_motor(168,143);//右旋回
        delay(750);//n秒の時n*1000の値を入力

        arm.write(90);//アームが前向きになるよう維持

        set_motor(65,151);//後退
        delay(100);//n秒の時n*1000の値を入力

        a = analogRead(A5);

        if(a>350){
          arm.write(90);//アームが前向きになるよう維持
          set_motor(168,143);//右旋回
          delay(1650);//n秒の時n*1000の値を入力
          set_motor(65,151);//後退
          delay(100);//n秒の時n*1000の値を入力
          a = analogRead(A5);

          if(a>400){
            arm.write(90);//アームが前向きになるよう維持
            set_motor(168,143);//右旋回
            delay(750);//n秒の時n*1000の値を入力
            set_motor(89,89);//停止
            delay(10000);//n秒の時n*1000の値を入力
          }
        }
      }else{//それ以外のとき直進するように速度をあたえる
        set_motor(120,50);//直進
        arm.write(90);//アームが前向きになるよう維持
      }
        delay(100);
      }

    \end{verbatim}

\end{document}
